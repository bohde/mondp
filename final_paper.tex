%final_paper.tex - Final Paper Draft for CS448
%author: Josh Bohde

\documentclass[11pt, oneside, notitlepage, final]{article}
\usepackage{algorithm}
\usepackage{algorithmic}
\usepackage{graphicx}
\DeclareGraphicsExtensions{.pdf,.png,.jpg}
\newcommand{\keywords}[1]{\par\addvspace\baselineskip
\noindent\enspace\ignorespaces\centering{\bf Keywords}\\#1}

\begin{document}


\title{Optimizing Urban Traffic Network Design:\\ A Multi-Objective Evolutionary Algorithm Approach.}
\author{Josh Bohde, BS in Computer Science}
\date{\today}
\maketitle

\begin{abstract}
A method is presented for solving both the discrete and continuous network design problem in a multiobjective manner with respect to urban traffic. A framework is presented using a multiobjective evolutionary algorithm and the traffic simulator Simulation for Urban Mobility (SUMO), using custom genetic operators. A sample network is optimized for both its discrete and continuous aspects, with comparisons drawn between the respective performance of the two. This framework is then applied to real world data drawn from a dataset of Cologne, Germany. These results are then used to show the feasibility of usage of the presented framework in a real world scenario.
\keywords{Network Design Problem, Evolutionary Algorithm, Multiobjective}
\end{abstract}

\clearpage
\part{Introduction}
    As urban areas grow, the aging road network is no longer able to handle the amount of traffic generated by the population, resulting in congestion and delays. These delays translate into a direct economic cost which is rising. A study of urban congestion in the United States  shows that the congestion costs rose from \$45.5 billion in 1995 to \$78.2 billion in 2005, adjusted for inflation \cite{7}. This congestion also resulted in 2.9 billion gallons of wasted fuel, and 4.2 billion hours in delays \cite{7}. The proposed research aims to limit this congestion, while balancing a variety of limiting factors to the network. The economic burden placed on the tax payers will be taken into account, as well as car emissions and the likelihood of accidents,  with the goal of minimizing these. A multi-objective approach will present a set of solutions that do not dominate each other, allowing for city planners to choose the most appropriate one, given their constraints.

    The investigator will develop a method for optimizing the design of urban traffic networks. Important to urban traffic networks are the secondary characteristics of the network, such as the cost of the improvements to the network. Previous methods for solving this problem have involved applying weights to the values of these characteristics in ranking possible solutions. The proposed method will instead use a multi-objective evolutionary algorithm (MOEA) to provide a Pareto Optimal set of urban traffic networks. The objectives of this algorithm will be to maximize performance of the network, minimize cost of changes to the network, minimize emissions, and minimize accidents.

    The investigator aims to improve upon previous related research in three ways. First, by using a multi-objective approach to optimize with respect to overall network performance and cost. Second, by using a traffic simulator in the fitness function in the evolutionary algorithm in order to provide fitnesses that are more applicable to real world situations. Third, scalability testing in order to demonstrate the feasibility of the proposed method in real world applications.

    The inclusion of a traffic simulator presents a challenge in the form of high computational costs. This necessitates investigation into representations of road networks and respective genetic operators that generate valid solutions, while minimizing total evaluations.


\part{Background}
    Urban traffic network design is a version of the network design problem (NDP). This problem is maximizing the amount of traffic on a network while minimizing the cost of the network, between a set of nodes representing places traffic can start or end, and links representing connections between nodes. There are two aspects to optimizing the performance of a network, discrete network design problem (DNDP) and continuous network design problem (CNDP). DNDP optimization is done by altering the discrete variables of the network, which is adding and subtracting links between nodes, which corresponds to adding roads and highways. CNDP optimization is achieved by altering the continuous variables, such as traffic light timings, speed limits, and number of lanes.

    Because the number of links is exponential with regards to the number of nodes, the search space for DNDP is very large. With the inclusion of CNDP, the search space is even larger. The search space is not simple, as evidenced by Braess's paradox, which shows that adding capacity to a network can decrease the overall performance of the network. It was shown that for a random graph, the probability of Braess's paradox occurring is high\cite{6}. This implies that when using stochastic algorithms to solve NDP, the search space is complex.

    In a paper on urban road quality, the authors identify several other factors that affect the quality of the road network for the city inhabitants, and several restrictions for traffic network design \cite{5}. Two objectives presented are the minimization of emissions and accidents. The effects of emissions are lowered as a result of lower speeds and traffic volume, with space between traffic and distance from residential areas also playing a part. Traffic accidents are less numerous with less traffic volume, and are less severe with lower speeds.

\part{Related Work}
    Various search algorithms have been used in traffic network design.  Search heuristers were used to  improve the Sioux Falls traffic network, a CNDP, using a bi-level programming model and simulated annealing \cite{1}. Fitness was determined by the travel cost between each node in the graph. Genetic algorithms were shown to be optimize for CNDP \cite{4}, using a case study of network with 4 nodes. This algorithm employed a deterministic user traffic assignment to establish fitness.  Genetic algorithms were shown to be comparable to simulated annealing for optimization of a CNDP using a real traffic network \cite{2} These were also tested against hill climbing and tabu search heuristics.

    Another paper presented a new formulation of the network design emphasizing robustness of the produced system in order to account for varying demands between nodes in the graph \cite{3}. This formulation of the problem and the resulting genetic algorithm was shown to produce high quality solutions on example problems.

    A MOEA is a type of evolutionary algorithm to optimize for one or more fitness values without reducing them to a single value. One solution is said to dominate another if its respective fitness values are no less than the other's, and at least one value greater than the other's respective value \cite{12}. Instead of a single solution, a set of non-domination solutions are returned, called the Pareto-optimal front.

    While there are many different formulations of MOEAs, a common problem of them is to have a large Pareto-optimal front, especially as the number of fitness values increases. Ɛ-dominance limits the Pareto-optimal front by not allowing solutions that dominate each other less than Ɛ to be included within the Pareto-optimal front \cite{11}. Conceptually this divides the area of possibly accepted fitness into a grid, only allowing one solution in each grid.
\part{Methodology}
    \section{Evolutionary Algorithm}
        The proposed MOEA is the Ɛ-domination based multi-objective algorithm (Ɛ-MOEA), as described in \cite{10}. It was chosen based upon its property of ensuring diversity amongst the Pareto-optimal front, which is an important quality for this problem. From a parameter perspective, Ɛ-MOEA is much like a standard MOEA, with the exception of its offspring size and Ɛ values. Because Ɛ-MOEA is a steady state algorithm, there is no need for an offspring size parameter. The algorithm also differs in the need for Ɛ values for each objective.


    \section{Network Representation}
        The network is represented as a graph: a set of vertices and edges, with each edge connecting two vertices. The addition and deletion of edges from the network satisfies discrete aspect of the NDP.

        To satisfy the continuous aspect, each edge has several properties affecting the performance of the individual edge. These include the shape of the edge, max speed, number of lanes, edge priority, and the direction of the edge.The priority of an edge determines how likely cars traveling on it are going to yield to cars on some intersecting edge. The max speed is the maximum speed that cars are allowed to travel. The number of lanes determines how many cars can be at the same progression on the edge. The direction of the edge determines whether or not the edge road on it is one way. Finally the shape of the edge determines the placement of the edge on the simulated area. This is represented as a set of points, consisting of floating points. The actual path of the road must visit each of these points in order before reaching the end point.


        %Formulas for these.



    \section{Genetic Operators}
        \subsection{Recombination}
            Recombination of individuals is achieved by through a process analogous to uniform crossover for bit string representations. For each edge in both parents, the edge from the first parent is assigned to a randomly chosen child, with the remaining edge going to the remaining child. For each edge not in both parents, it is randomly assigned to a randomly chosen child. The exact procedure is given in Algorithm \ref{recomb}.

            \begin{algorithm}
            \caption{Network Recombination}
            \label{recomb}
            \begin{algorithmic}
            \FORALL{$edge \in parent_1\_edges \cup parent_2\_edges$}
            \STATE Pick a random integer $r$ uniformly from $[0,1]$
            \IF{$edge \notin parent_2\_edges$}
            \STATE $child_r\_edges \Leftarrow child_r\_edges \cap edge$ 
            \ELSE
            \STATE $child_{(1-r)}\_edges \Leftarrow child_{(1-r)}\_edges \cap$ ($edge$ in $parent_1\_edges$)
            \STATE $child_r\_edges \Leftarrow child_r\_edges \cap$ ($edge$ in $parent_2\_edges$)
            \ENDIF
            \ENDFOR
            \end{algorithmic}
            \end{algorithm}

        \subsection{Mutation}
            
            \subsubsection{Discrete}
            Mutation of networks is achieved by randomly adding or deleting edges. For every child generated, it is mutated one time. The exact mutation procedure is described in Algorithm \ref{mut:disc}.

            \begin{algorithm}
            \caption{Discrete Mutation}
            \label{mut:disc}
            \begin{algorithmic}
            \STATE $n \Leftarrow $ number of elements in $vertices$
            \STATE Pick two random integers $j, k$ uniformly from $[0...n]$ where $j \neq k$
            \STATE Generate an edge $e$ connecting $vertices_j$ and $vertices_k$
            \IF{$e \notin edges$}
            \STATE $edges \Leftarrow edges \cap e$
            \ELSE
            \STATE $edges \Leftarrow edges - e$
            \ENDIF
            \end{algorithmic}
            \end{algorithm}

            \subsubsection{Continuous}
            Continuous mutation is done by altering individual edges. The aspects modified are edge shape, number of lanes, lane spread, and edge priority.
            Edge shape is modified by taking the current edge shape, choosing a point on it, and randomly moving it. For edges in the original graph, edge shape cannot be modified. For other edges, the chance to mutate this is user defined. This algorithm is described in Algorithm \ref{mut:edge}.

            \begin{algorithm}
            \caption{Edge Shape Mutation}
            \label{mut:edge}
            \begin{algorithmic}
            \STATE $shape$ is the ordered set of points making up the shape of the edge.
            \STATE Pick a random real number $r$ uniformly from $[0, 1]$.
            \IF{$r < chance\_to\_mutate$}
            \STATE Pick a random integer $n$ uniformly from $[1, n-1]$ where $n$ is the number of elements in $shape$
            \STATE $H \Leftarrow$ the hyperbolic function created by $shape_n$ and $shape_{(n+1)}$
            \STATE Pick a random real number $t$ uniformly from $[0, 1]$.
            \STATE $p \Leftarrow H(t)$
            \STATE Pick two random real numbers $x_r, y_r$ uniformly from $[0, 1]$.
            \STATE $p \Leftarrow p + (x_r * max\_delta\_x, y_r * max\_delta\_y)$
            \STATE Insert $p$ into $shape$ between $shape_n$ and $shape_{(n+1)}$
            \ENDIF
            \end{algorithmic}
            \end{algorithm}

            With a user-defined chance, the number of lanes is mutated according by adding or subtracting a random number of lanes, up to a user-defined maximum (max delta). For edges in the original graph, this cannot be altered. The algorithm is described in Algorithm \ref{mut:nolanes}.

            \begin{algorithm}
            \caption{Lane Number Mutation}
            \label{mut:nolanes}
            \begin{algorithmic}
            \STATE Pick a random real number $r$ uniformly from $[0, 1]$.
            \IF{$r < chance\_to\_mutate$}
            \STATE Pick a random integer $l$ uniformly from $[-max\_delta, max\_delta]$
            \STATE $lanes \Leftarrow lanes + l$
            \STATE $lanes \Leftarrow max(1, lanes)$
            \ENDIF
            \end{algorithmic}
            \end{algorithm}

            With a user-defined chance, the lane spread is mutated by randomly choosing one of the three possibilities, right, center, or left. The algorithm is described in Algorithm \ref{mut:spread}.

            \begin{algorithm}
            \caption{Lane Spread Mutation}
            \label{mut:spread}
            \begin{algorithmic}
            \STATE Pick a random real number $r$ uniformly from $[0, 1]$.
            \IF{$r < chance\_to\_mutate$}
            \STATE Pick a random integer $s$ uniformly from $[-1, 1]$
            \STATE $spread \Leftarrow s$
            \ENDIF
            \end{algorithmic}
            \end{algorithm}


            With a user-defined chance, the number of lanes is mutated according by adding or subtracting an a randomly generated number to the edge's priority, up to a user-defined maximum (max delta).The algorithm is described in Algorithm \ref{mut:prior}.
            \begin{algorithm}
            \caption{Edge Priority Mutation}
            \label{mut:prior}
            \begin{algorithmic}
            \STATE Pick a random real number $r$ uniformly from $[0, 1]$.
            \IF{$r < chance\_to\_mutate$}
            \STATE Pick a random integer $p$ uniformly from $[-max\_delta, max\_delta]$
            \STATE $priority \Leftarrow priority + p$
            \STATE $priority \Leftarrow max(0, priority)$
            \ENDIF
            \end{algorithmic}
            \end{algorithm}



    \section{Fitness Measure}
        The fitness is a tuple, consisting of the network performance and cost. 

        \subsection{Network Performance}
            Common to all of the previous research is the usage of mathematical models to emulate the traffic of the network. While this is useful for determining the theoretical maximum performance of a network, it may not accurately reflect the performance of a real urban network. To better model the performance of a real network, a traffic simulator is proposed to generate the characteristics of the network from which the fitness will be derived. The planned traffic simulator is Simulation of Urban Mobility (SUMO) [8].  SUMO is a microscopic, space-continuous traffic simulator, developed by the Institute of Transport Research at the German Aerospace Centre. Its microscopic model emulates traffic on a per-car basis, while keeping computation time low. This simulator will provide details of network performance, vehicle emissions, and network safety.
            In order to test the performance in the simulation, the user must supply a flow file, describing a set of flows, composed of a start edge, and ending edge, a beginning and end time, and the number of times to repeat the route. From this, the program can generate specific routes for evaluating the network.
            Network performance is measured by compiling the edges generated by the algorithm into a traffic network, and running the simulation. From the simulation output, the average time for a car to reach its destination journey will be measured. This is used as the measure for network performance.

            A flowchart is shown in Figure \ref{fig:sumo}, detailing the exact procedure on network evaluation.

            \begin{figure}[h!]
            \label{fig:sumo}
            \centering
            \includegraphics[scale=0.5]{simulation_toolchain}
            \caption{Flowchart for Network Performance Evaluation}
            \end{figure}

        \subsection{Network Cost}
            Network cost is the sum of the land cost, building costs, and intersection costs of the network.
            \subsubsection{Land Cost}

                Land cost is determined by a supplied to the algorithm, giving the prices for land at a given interval. Using a ray tracing algorithm, each edge produces a corresponding inhabitance matrix. This algorithm is described in Algorithm \ref{cost:edge}.

                \begin{algorithm}
                \caption{Edge Intersect Matrix}
                \label{cost:edge}
                \begin{algorithmic}
                \STATE $I_{m \times n}$ is the intersect matrix, initialized as the zero matrix.
                \STATE $t \Leftarrow 0$
                \STATE $j$ is the initial x value of the first point in $I$.
                \STATE $k$ is the initial y value of the first point in $I$.
                \STATE $I_{j, k} \Leftarrow 1$
                \WHILE{$t < 1$}
                \STATE $t_y$ is the distance along the $y$-axis to the next element in the matrix, along the edge.
                \STATE $t_x$ is the distance along the $x$-axis to the next element in the matrix, along the edge.
                \IF{$t_y < t_x$}
                \STATE $t \Leftarrow t + t_y$
                \STATE $k \Leftarrow k + 1$
                \ELSE
                \STATE $t \Leftarrow t + t_x$
                \STATE $j \Leftarrow j + 1$
                \ENDIF
                \STATE $I_{j, k} \Leftarrow 1$
                \ENDWHILE
                \end{algorithmic}
                \end{algorithm}

                For each edge in the network, a network inhabitance matrix is produced, where each element is the bitwise or of the corresponding elements in each edge inhabitance matrix. This algorithm is described in Algorithm \ref{cost:edges}.

                \begin{algorithm}
                \caption{Network Intersect Matrix}
                \label{cost:edges}
                \begin{algorithmic}
                \STATE $I_{m \times n}$ is the intersect matrix, initialized as the zero matrix.
                \FORALL{$i \in edge\_intercept\_matrices$}
                \FOR{$j$ = 1 to $m$}
                \FOR{$k$ = 1 to $n$}
                \STATE $I_{j, k} \Leftarrow max(0, i_{j,k})$
                \ENDFOR
                \ENDFOR
                \ENDFOR
                \end{algorithmic}
                \end{algorithm}

                The overall land cost is determined by summing the multiplication of corresponding elements of the cost matrix and network inhabitance matrix. This algorithm is described in Algorithm \ref{cost:land}

                \begin{algorithm}
                \caption{Network Land Cost}
                \label{cost:land}
                \begin{algorithmic}
                \STATE $I_{m \times n}$ is the intersect matrix, containing the network intercepts
                \STATE $C_{m \times n}$ is land cost matrix
                \STATE $r$ = 0
                \FOR{$j$ = 1 to $m$}
                \FOR{$k$ = 1 to $n$}
                \STATE $r \Leftarrow r + I_{j, k} \times C_{j, k}$
                \ENDFOR
                \ENDFOR
                \RETURN $r$
                \end{algorithmic}
                \end{algorithm}
                

            \subsubsection{Building Costs}
                Building costs require two user set parameters, the cost of road per unit per lane and cost of intersection. The building cost is
                the sum of all costs of intersections and the sum of the road costs for each edge. The number of intersections are determined when establishing the edge intersection matrices. The road costs for each edge is determined by finding the lengths edge, and multiplying that by the number of lanes in the edge, and the cost of road per unit per lane.


\part{Experimental Setup}
    \setcounter{section}{0}
    \section{Comparison of Discrete and Continuous Network Design Problem}
        The goal of this experiment  was to determine whether optimizing for the CNDP as opposed to the DNDP provided a benefit in the quality of solutions contained in the Pareto-optimal front that offset the added complexity to the search space. To do so, the algorithm was run on a randomly generate dataset, with the continuous mutations turned off for one, and on for another. In Table \ref{tab:ex1a}, the values common to both tested algorithms are given. Values for the specific algorithms are in Table \ref{tab:ex1b}.
 
        \begin{table}[H]
        \label{tab:ex1a}
        \centering
        \begin{tabular}{ | c | c | c | c |}
            \hline
            Number of Runs  & Evaluations   & Initial Population    & Fitness Epsilons  \\ \hline
            30              & 1000          & 50                    & (.1, .1, .1)          \\ \hline
        \end{tabular}
        \caption{Common Experiment Values}
        \end{table}

        \begin{table}[H]
        \label{tab:ex1b}
        \centering
        \begin{tabular}{ | c | c | c |}
            \hline
            Attribute           &       Continuous  &   Discrete    \\ \hline
            Shape Mutate Chance &       5\%         &   0\%         \\ \hline
            Shape Mutate Delta  &       1           &   0           \\ \hline
            Priority Mutate Chance &    5\%         &   0\%         \\ \hline
            Priority Mutate Delta &     1           &   0           \\ \hline
            Lane Mutate Chance    &     5\%         &   0\%         \\ \hline
            Lane Mutate Delta     &     2 &             0           \\ \hline
            Spread Mutate Chance  &     1\%         &   0\%         \\ \hline
        \end{tabular}
        \caption{Individual Experiment Values}
        \end{table}


    \section{Testing of Continuous Network Design Problem on Real World Data}
        In order to demonstrate the feasibility of using the algorithm for real world applications, it was tested on a subset of the freely available data of Cologne, Germany. The algorithm specific values used for testing are shown in Table \ref{tab:ex2}.

        \begin{table}[H]
        \label{tab:ex2}
        \centering
        \begin{tabular}{ | c | c |}
            \hline
            Runs                &       5           \\ \hline
            Initial Population  &       50          \\ \hline
            Evaluations         &       1000        \\ \hline
            Fitness Epsilons    &       (.1, .1, .1)\\ \hline
            Shape Mutate Chance &       5\%         \\ \hline
            Shape Mutate Delta  &       1           \\ \hline
            Priority Mutate Chance &    5\%         \\ \hline
            Priority Mutate Delta &     1           \\ \hline
            Lane Mutate Chance    &     5\%         \\ \hline
            Lane Mutate Delta     &     2          \\ \hline
            Spread Mutate Chance  &     1\%         \\ \hline
        \end{tabular}
        \caption{Individual Experiment Values}
        \end{table}


\part{Results}
    \setcounter{section}{0}
    \section{Comparison of Discrete and Continuous Network Design Problem}
        The final nondominating set from all of the runs for the continous experiment is shown in Table \ref{tab:res1a}, with the set from the discrete experiment in Table \ref{tab:res1b}.

    \begin{table}[H]
        \label{tab:res1a}
        \centering
        \begin{tabular}{ | c | c | c | }
            \hline
            17.0 & 0.0 & 107998.690502 \\ \hline
            16.0 & 0.0 & 243551.39066 \\ \hline
            18.0 & 0.0 & 23350.3471944 \\ \hline
        \end{tabular}
        \caption{Continuous Nondominated Set}
    \end{table}

    \begin{table}[H]
        \label{tab:res1b}
        \centering
        \begin{tabular}{ | c | c | c | }
            \hline
            19.0 & 1.5 & 21944.0 \\ \hline
            24.67 & 1.0 & 23749.0 \\ \hline
        \end{tabular}
        \caption{Discrete Nondominated Set}
    \end{table}

    A graph showing all the nondominated individuals over all the runs from the continuous dataset are shown in Figure \ref{fig:res1b}. A similar graph is shown in Figure \ref{fig:res1a}.

    \begin{figure}[h!]
    \label{fig:res1b}
    \centering
    \includegraphics[scale=0.75]{cont_graph}
    \caption{Continuous Nondominated Individuals}
    \end{figure}

    \begin{figure}[h!]
    \label{fig:res1a}
    \centering
    \includegraphics[scale=0.75]{discrete}
    \caption{Discrete Nondominated Individuals}
    \end{figure}

    \section{Testing of Continuous Network Design Problem on Real World Data}
        Forthcoming.

\part{Discussion}
    \setcounter{section}{0}
    \section{Comparison of Discrete and Continuous Network Design Problem}
        The results presented show that it the given method does provide a set of nondominating solutions representing an improvement over the original urban network, for both the DNDP and CNDP. 

        With the results presented demonstrate that solving for the continuous aspects of the network can provide better solutions than solving for the discrete aspect alone. This possibility is gained at the expense of an expanded search space. The most notable of the improvements that the CNDP algorithm offered over the DNDP algorithm is that the CNDP was able to eliminate wait time by shifting avenues of traffic away from each other, preventing intersections, which results in stops. 

    \section{Testing of Continuous Network Design Problem on Real World Data}
        Forthcoming.

\part{Conclusion}
    The framework presented here is an effective method for solving both the DNDP and CNDP for urban traffic networks, while presenting the user with more options than producing a weighting function and a single objective approach would afford. Ultimately, this algorithm can assist city planners with designing urban areas. While the computation time for the simulation can be quite high, the time frame for such projects should allow for adequate computation time.

\part{Future Work}
    The framework should be expaned to handle more fitness measures, adding to its usefulness for city planners. The amount of pollution caused by traffic and how prone to accidents the roadway is should be a priority.
    To expand upon this framework's application in real world situations, a method should be devised to allow for two way dialog between the user and the program, allowing the program to suggest upgrades that can be made now, while the user can update the progress of the corresponding real network. 

\clearpage
\bibliographystyle{plain}
\bibliography{final_paper.bib}
\clearpage
\appendix
\part*{Appendices}
\section{Code Repository}
$http://github.com/joshbohde/mondp/$
\end{document}