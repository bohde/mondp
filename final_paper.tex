%final_paper.tex - Final Paper Draft for CS448
%author: Josh Bohde

\documentclass[11pt, oneside, notitlepage, draft]{article}
\usepackage{algorithmic}
\newcommand{\keywords}[1]{\par\addvspace\baselineskip
\noindent\enspace\ignorespaces\centering{\bf Keywords}\\#1}

\begin{document}


\title{Optimizing Urban Traffic Network Design:\\ A Multi-Objective Evolutionary Algorithm Approach.}
\author{Josh Bohde, BS in Computer Science}
\date{\today}
\maketitle

\begin{abstract}
This is my abstract. It is very wordy. Words words words.
\keywords{Network Design Problem, Evolutionary Algorithm, Multiobjective}
\end{abstract}

\clearpage
\section{Introduction}
    The investigator will develop a method for optimizing the design of urban traffic networks. Important to urban traffic networks are the secondary characteristics of the network, such as the cost of the improvements to the network. Previous methods for solving this problem have involved applying weights to the values of these characteristics in ranking possible solutions. The proposed method will instead use a multi-objective evolutionary algorithm (MOEA) to provide a Paretor Optimal set of urban traffic networks. The objectives of this algorithm will be to maximize performance of the network, minimize cost of changes to the network, minimize emissions, and minimize accidents.

    The investigator aims to improve upon previous related research in three ways. First, by using a multi-objective approach to optimize with respect to overall network performance, cost, emissions, and traffic safety. Second, by using a traffic simulator in the fitness function in the evolutionary algorithm in order to provide fitnesses that are more applicable to real world situations. Third, scalability testing in order to demonstrate the feasibility of the proposed method in real world applications.

    The inclusion of a traffic simulator presents a challenge in the form of high computational costs. This necessitates investigation into representations of road networks and respective genetic operators that generate valid solutions, while minimizing total evaluations.

    As urban areas grow, the aging road network is no longer able to handle the amount of traffic generated by the population, resulting in congestion and delays. These delays translate into a direct economic cost which is rising. In study of urban congestion in the United States [7] shows that the congestion costs rose from \$45.5 billion in 1995 to \$78.2 billion in 2005, adjusted for inflation. This congestion also resulted in 2.9 billion gallons of wasted fuel, and 4.2 billion hours in delays [7]. The proposed research aims to limit this congestion, while balancing a variety of limiting factors to the network. The economic burden placed on the tax payers will be taken into account, as well as car emissions and the likelihood of accidents,  with the goal of minimizing these. A multi-objective approach will present a set of solutions that do not dominate each other, allowing for city planners to choose the most appropriate one, given their constraints.


\section{Background}
    Urban traffic network design is a version of the network design problem (NDP). This problem is maximizing the amount of traffic on a network while minimizing the cost of the network, between a set of nodes  representing places traffic can start or end, and links representing connections between nodes. There are two aspects to optimizing the performance of a network, discrete network design problem (DNDP) and continuous network design problem (CNDP). DNDP optimization is done by altering the discrete variables of the network, which is adding and subtracting links between nodes, which corresponds to adding roads and highways. CNDP optimization is achieved by altering the continuous variables, such as traffic light timings, speed limits, and number of lanes.

    Because the number of links is exponential with regards to the number of nodes, the search space for DNDP is very large. With the inclusion of CNDP, the search space is even larger. The search space is not simple, as evidenced by Braess's paradox, which shows that adding capacity to a network can decrease the overall performance of the network. In [6], it was shown that for a random graph, the probability of Braess's paradox occurring is high. This implies that when using stochastic algorithms to solve NDP, the search space is complex.

    In [5], the authors identify several other factors that affect the quality of the road network for the city inhabitants, and several restrictions for traffic network design. Two objectives presented are the minimization of emissions and accidents. The effects of emissions are lowered as a result of lower speeds and traffic volume, with space between traffic and distance from residential areas also playing a part. Traffic accidents are less numerous with less traffic volume, and are less severe with lower speeds.

\section{Related Work}
    Various search algorithms have been used in traffic network design.  In [1], the authors improve the Sioux Falls traffic network, a CNDP, using a bi-level programming model and simulated annealing. Fitness was determined by the travel cost between each node in the graph. Genetic algorithms were shown to be optimize for CNDP in [4], using a case study of network with 4 nodes. This algorithm employed a deterministic user traffic assignment to establish fitness.  Genetic algorithms were shown to be comparable to simulated annealing for optimization of a CNDP using a real traffic network [2]. These were also tested against hill climbing and tabu search heuristics.

    In [3], the authors presented a new formulation of the network design emphasizing robustness of the produced system in order to account for varying demands between nodes in the graph. This formulation of the problem and the resulting genetic algorithm was shown to produce high quality solutions on example problems.

    A MOEA is a type of evolutionary algorithm to optimize for one or more fitness values without reducing them to a single value. As described in [12], one solution dominates another if its respective fitness values are no less than the other's, and at least one value greater than the other's respective value. Instead of a single solution, a set of non-domination solutions are returned, called the Pareto-optimal front.

    While there are many different formulations of MOEAs, a common problem of them is to have a large Pareto-optimal front, especially as the number of fitness values increases. The concept of  Ɛ-dominance was introduced in [11], which limits the Pareto-optimal front by not allowing solutions that dominate each other less than Ɛ to be included within the Pareto-optimal front. Conceptually this divides the area of possibly accepted fitness into a grid, only allowing one solution in each grid.
\section{Methodology}
    \subsection{Network Representation}
        The network is represented as a graph: a set of vertices and edges, with each edge connecting two vertices. The addition and deletion of edges from the network satisfies discrete aspect of the NDP.

        To satisfy the continuous aspect, each edge has several properties affecting the performance of the individual edge. These include the shape of the edge, max speed, number of lanes, lane priority, and the direction of the edge.

        %Formulas for these.

    \subsection{Genetic Operators}
        \subsubsection{Recombination}
            Recombination of individuals is achieved by through a process analagous to crossover.
            \begin{algorithmic}
            \FORALL{$edge$ in $G$}
            \STATE $i \leftarrow i + 1$
            \ENDFOR
            \end{algorithmic}

        \subsubsection{Mutation}
            Mutation of networks is achieved by randomly adding or deleting nodes. 
    \subsection{Fitness Measure}
        The fitness is a tuple, consisting of the network performance and cost. 

        \subsubsection{Network Performance}
            The performance of the network is determined by the usage of SUMO, a traffic simulator.
        \subsubsection{Network Cost}



\section{Experimental Setup}

\section{Results}

\section{Discussion}

\section{Conclusion}

\section{Future Work}

\clearpage
\appendix
\part*{Appendices}
\section{Code Repository}
{\emph http://github.com/joshbohde/mondp/}
\end{document}